\documentclass[letter, 12pt]{article} %
\usepackage{graphicx,amssymb} %
\usepackage{hyperref}
\hypersetup{
	colorlinks	=	true,
	linkcolor	=	blue,
	filecolor	=	magenta,      
	urlcolor	=	blue,
	citecolor	=	blue
}

\textwidth=16cm \hoffset=-1.2cm %
\textheight=25cm \voffset=-2cm %

\pagestyle{empty} %

\date{} %

%\def\keywords#1{{\bf Keywords: } {#1}} %

\setlength{\parskip}{0.5em}

\setlength{\parindent}{0em}
% Please, do not change any of the above lines


\begin{document}

% Type down your paper title
\title{Ground-level Particulate Matter Mass and Component Observation Imputation and Correction using Remote-Sensing}

% Authors
\author{Wesley Burr* (presenting author) \\ %
       Trent University, Canada \\
       \url{wesleyburr@trentu.ca}\\
       \url{http://www.wesleyburr.com} \\ \\% Affiliation 1
       Hwashin H.~Shin \\ % If any other author with different Affilation
       EHSRB, Health Canada, Ottawa, Canada, and \\ 
       Dept.~of Mathematics \& Statistics, Queen's University, Canada \\
       \url{hshs@queensu.ca}\\% Affiliation 2 (if needed)
       }%

\maketitle

%\thispagestyle{empty}

% The abstract
\begin{abstract}
A country-wide Canadian study on the interactions between particulate matter
and human health effects is currently ongoing under the 
guidance and funding of Health Canada and Environment and Climate Change
Canada. As part of this study, we are imputing and error-correcting a
large-scale database of hourly, daily, and monthly particulate matter
concentration measurements. In this talk we will discuss the use of
remote sensing concentration observations (satellite) as baseline and
comparison observations for the imputation and correction of ground-level
particulate matter mass and component observations. The differing time
and geographic scales for observation make this an interesting time 
series and spectrum estimation problem, with a number of powerful applications.
\end{abstract}

\vspace{.5cm}

{\bf Keywords: }{TIES2018, time series, particulate matter, remote sensing, imputation, interpolation} % Write down at least 3 Keywords

\end{document}
